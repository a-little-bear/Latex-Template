\documentclass[12pt]{alittlebear}
\def\name{Joseph Siu}
\def\course{MAT100: Introduction to Mathematical Latexing}
\def\headername{Homework }
\def\headernum{114514}

\begin{document}

\newe[Trivial Exercise]{e1}{
    \newn{
        This is an example document.
    }
    \newq{q1}{
        \[1+1\?3.\]
        \qbreak
        \newh{
            Consider $1+1=2$.
        }
        \envbreak
        \newcl{clm1}{
            Yes.
            \qbreak
            \newp{[Proof of Claim \ref{claim:clm1}]
                This is a good question.
            }
        }
        \newp{[Proof by Triviality]
            Trivial.
        }
        \newr{
            This is very trivial.
        }
    }
    \envbreak
    \newt[HelloWorld]{thm1}{
        This is NOT a theorem.
    }
    \newd[Byebye]{def1}{
        This is DEFINITELY a theorem.
    }
}

\newch[Non-Trivial Chapter]{chp1}{ 
    \newn{
        \quad Recall that we have already explained in lecture that all rational function of a single variable can be integrated in finite terms. Starting from there, we introduce integrals of the form, known as the \textbf{binomial integrals}:
        \[\begin{aligned}J_{p,q}=\int(a+bz)^pz^q\D z,a,b\in\mathbb{R},p,q\in\mathbb{Q}. \end{aligned} \numberthis \label{eq1}\]
    
        This exercise aims at studying the rationalization of the binomial integral, as well as some of its applications.
        }
    \retcbcnt{question}
    \newq{c1q1}{
        Assume that $p\in\Z$, rationalise the integrand.
       \qbreak
       \newn{
            Since $q\in\Q$, this implies there exists $m\in\Z, n\in\N$ such that $q=\frac{m}{n}$. Then, $J_{p,q}$ can be written as \[\begin{aligned}J_{p,q}=\int(a+bz)^pz^{\frac{m}{n}}\D z.\end{aligned}\] Let $z=x^n$, then $\D z=n\cd x^{n-1}\D x$. Replace $z$ with $x$ we have \[\begin{aligned}J_{p,q}=\int(a+b\cd x^n)^p \cd (x^n)^{\frac{m}{n}}\cd n\cd x^{n-1}\D x.\end{aligned}\] Simplify it and we get \[\begin{aligned}J_{p,q}=n\int(a+b\cd x^n)^p \cd x^{m+n-1}\D x.\end{aligned}\] Now, by binomial theorem we can see the integrand is rationalised, as needed.
       }
    }
}

\end{document}