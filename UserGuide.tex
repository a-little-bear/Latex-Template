\documentclass[12pt]{alittlebear}

\def\name{Joseph Siu}
\def\course{MAT114: Latex IV}
\def\headername{UserGuide }
\def\headernum{XX}

\begin{document} 

\tableofcontents

\section{Introduction}

\noindent
The DocumentClass \textit{alittlebear.cls} aims to provide a simple and easy template for writing math notes on latex. The link to alittlebear.cls, Example.tex, and Template.tex can be found at \url{https://github.com/a-little-bear/Latex-Template}.\\

\noindent
Here is the template of a new tex file: 
\begin{note}
    \begin{verbatim}
\documentclass[12pt]{alittlebear}

\def\name{}
\def\course{}
\def\headername{}
\def\headernum{}

\begin{document} 



\end{document}    
    \end{verbatim}
\end{note}

\newpage

\section{Settings}

\subsection{Document Preamble}\hfill

\verb=\documentclass[#pt]{alittlebear}=
\explain{\#: font size in $\{10, 11, 12\}$, in pt}

\verb=\def\name{#}=
\explain{\#: author name}

\verb=\def\course{#}=
\explain{\#: course name}

\verb=\def\headername{#}=
\explain{\#: header name}

\verb=\def\headernum{#}=
\explain{\#: header number}



\subsection{Document Class}\hfill

\verb=\setlength\parindent{0pt}=
\explain{Indentation, Default: 0pt}

\verb=\def\maincolor{brown}=
\explain{Main color, Default: brown}

\begin{verbatim}
  \RequirePackage[margin=0.5in]{geometry}
\end{verbatim}
\ind[20]{Margin, Default: 0.5in, may change to 1in}

\verb=\theoremstyle{definition}=
\explain{Theorem style from asmthm, Default: definition}






\newpage
\section{Commands}

\subsection{Formatting Commands}\hfill

\verb=\indenv[2][1]{\begin{adjustwidth}{#1cm}{}#2\end{adjustwidth}}=
\explain{Indented environment, use package \textit{changepage}}
\begin{verbatim}
                Example: \indenv{
                    This is an indented environment (multiple paragraphs)
                }
                \indenv[10]{
                    This is an indented environment with 10mm indentation
                }
\end{verbatim}

\verb=\ind[1][5]{\hspace*{#1mm}}=
\explain{\#1: indentation, Default: 5mm}
\begin{verbatim}
                Example: \ind{
                    This is an indented paragraph
                }
                \ind[10]{
                    This is an indented paragraph with 10mm indentation
                }
\end{verbatim}

\verb=\unind[1][5]{\hspace*{-#1mm}}=
\explain{\#1: unindentation, Default: -5mm}

\verb=\explain[2][20]{\\\ind[#1]{#2}}=
\explain{\#1: indentation, Default: 20mm, \#2: explanation}

\verb=\np{\newpage}=
\explain{shortcut for newpage command}

\verb=\ds{\displaystyle}=
\explain{shortcut for displaystyle command}

\newpage
\subsection{TColorBox Commands}\hfill

\verb=\qbreak=
\explain{End the question and follow by the proof / solution}

\verb=\envbreak=
\explain{And a seperator line within an environment}

\verb=\tcbcnt=
\explain{Set the counter for tcolorbox theorem environment}

\verb=\retcbcnt=
\explain{Reset the counter for tcolorbox theorem environment}

\verb=\newn,\newm=
\explain{New notitle tcolorbox}
\begin{verbatim}
                Example: \newn{
                    This is a new notitle "note" tcolorbox 
                }
                \newm{
                    This is a new notitle "mathnote" tcolorbox 
                }
\end{verbatim}

\verb=\newh,\newr,\newp=
\explain{New asmthm theorem tcolorbox environment with prefixes}
\begin{verbatim}
                Example: \newh{
                    This is a new "hint" asmthm tcolorbox environment 
                }
                \newr{
                    This is a new "remark" asmthm tcolorbox environment 
                }
                \newp{
                    This is a new "proof" asmthm tcolorbox environment 
                }
\end{verbatim}


\verb=\newq,\newcl,\newd,\newco,\newt,\newl,\newe,\newu,\newch=
\explain{New TColorBox theorem environment with titles}
\begin{verbatim}
                Example: \newq[optional: #EnvName]{#label}{
                    This is a new "question" tcolorbox theorem environment 
                }
                \newcl[optional: #EnvName]{#label}{
                    This is a new "claim" tcolorbox theorem environment 
                }
                \newd[optional: #EnvName]{#label}{"definition"}
                \newco[optional: #EnvName]{#label}{"corollary"}
                \newt[optional: #EnvName]{#label}{"theorem"}
                \newl[optional: #EnvName]{#label}{"lemma"}
                \newe[optional: #EnvName]{#label}{"exercise"}
                \newu[optional: #EnvName]{#label}{"unit"}
                \newch[optional: #EnvName]{#label}{"chapter"}
\end{verbatim}

\verb=\ref{#1:#label}=
\explain{Use ref to reference the environment, where \#1:\#label e.g. is "question:q1"}

\verb=\tbox[optional:#1]{#2}=
\explain{optional \#1 define more options, \#2 is the centered title}

\newpage
\subsection{Math Commands}\hfill

\verb=\numberthis=
\explain{Add the line number in unnumbered math environment}

\verb=\T[1]{\text{#1}}=
\explain{Abbreviation of $\backslash$text\{\}}

\begin{verbatim}
  \Al[3]{#1 &=#2 &\text{#3}&&\\}
\end{verbatim}
\ind[20]{(left) = (right) + (explanation)}

\verb=\cd{\cdot}=
\explain{Abbreviation of $\backslash$cdot}

\verb=\alt[1]{\intertext{#1}}=
\explain{Insert line between align math equations, $\backslash\backslash$ included}

\begin{verbatim}
  \bb[1]{\mathbb{#1}}
  \cal[1]{\mathcal{#1}}
  \sc[1]{\textsc{#1}}
\end{verbatim}
\ind[20]{More shortcuts for mathbb, mathcal, textsc}

\verb=\D{\mathop{}\!\mathrm{d}}=
\explain{d symbol for differentiation, example: $\backslash$D x}

\begin{verbatim}
  \DD[2]{\frac{\D #1 }{\D #2}}
\end{verbatim}
\ind[20]{Leibniz's notation of differentiation, example: $\backslash$DD\{x\}\{y\}}

\verb=\vspan{span}=
\explain{span in linear algebra, Math Operator}

\verb=\rank{rank}=
\explain{rank in linear algebra, Math Operator}

\verb=\im{im}=
\explain{image in linear algebra, Math Operator}

\verb=\sgn{sgn}=
\explain{defined math operator sgn as the sign function}



\newpage\subsection{Symbol Abbreviations}\hfill
\begin{verbatim}
  \C -> \mathbb{C} == complex
  \R -> \mathbb{R} == reals
  \Q -> \mathbb{Q} == rationals
  \Z -> \mathbb{Z} == integers
  \N -> \mathbb{N} == naturals 
  \F -> \mathbb{F} == field

  \al -> \alpha
  \be -> \beta
  \ga -> \gamma
  \Ga -> \Gamma
  \ep -> \varepsilon
  \de -> \delta
  \sig -> \sigma
  \Sig -> \Sigma
  \p -> \partial

  \? -> \stackrel{?}{=} == question mark on equal sign
  \ra -> \rightarrow == rightarrow (single line)
  \Ra -> \Rightarrow == Rightarrow (double lines)
  \is -> \equiv == equivalent (triple lines)
  \injective \surjective \bijective

  \arr = angle brackets
  \bra = parenthesis ()
  \sqrbra = square brackets []
  \curbra = curly brackets {}
  \abs = absolute value | |
  \norm = double absolute || ||
  \ceil = ceiling + () + ceiling
  \floor = floor + () + floor
  \near = floor + () + ceiling

  \func[3]{#1: #2 \rightarrow #3} == function (name, domain, codomain)
  \Pset{#} -> \mathcal{P}(#) == power set
  \Relate{#}{##} -> #\mathcal{R}## == relation
  \GF[1][2]{\bb{F}_{#1}} == Galois field, default #1 = 2
  \modulo[1][n]{\Z/#1\Z} == modulo, default #1 = n
  
  \P -> \mathbb{P} == primes
  \nil -> \varnothing == empty set
  \O -> \mathcal{O} == big O 
  \relate -> \mathcal{R} == relate (relation)

\end{verbatim}

\verb==
\explain{}

\verb==
\explain{}

\verb==
\explain{}

\verb==
\explain{}

\verb==
\explain{}


\newpage
\section{Other Notable Commands}

\begin{verbatim}
  \renewcommand{\qedsymbol}
  {$_{\scriptstyle \substack{\sc{quod}\\\sc{erat}\\\sc{dem}
  \scalebox{0.53}{$\blacksquare$}}}$}
\end{verbatim}
\ind[20]{Modified QED symbol}

\begin{verbatim}
  \lstnewenvironment{CPP}{\lstset{language=C++}}{}
\end{verbatim}
\ind[20]{New environment for C++ et al code listing}

\begin{verbatim}
                Example (CPP = C++, Cl = C):
                    \begin{CPP}
                        hi
                    \end{CPP}
\end{verbatim}


\newpage
\section{Known Bugs / Improvements}

\begin{enumerate}
    \item Nest chapter, exercise, unit together are unlickly to work.
    \item To improve readability, the environments should try to not be nested. 
    \item Extend listings and tikzpictures.
    \item The number counters cannot align with section numbers.
    \item The color box sometimes touches the footer.
    \item Make sure use \verb=\np= or \verb=\newpage= for new exercise/unit/chapter so that the page splitting functions properly.
\end{enumerate}



\end{document}